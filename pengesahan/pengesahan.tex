\begin{flushleft}
    % Ubah kalimat berikut sesuai dengan nama departemen dan fakultas
    \textbf{Departemen Teknik Komputer - FTEIC}\\
    \textbf{Institut Teknologi Sepuluh Nopember}\\
  \end{flushleft}
  
  \begin{center}
    % Ubah detail mata kuliah berikut sesuai dengan yang ditentukan oleh departemen
    \underline{\textbf{EC184701 - PRA TUGAS AKHIR (2 SKS)}}
  \end{center}
  
  \begin{adjustwidth}{-0.2cm}{}
    \begin{tabular}{lcp{0.7\linewidth}}
  
      % Ubah kalimat-kalimat berikut sesuai dengan nama dan NRP mahasiswa
      Nama Mahasiswa &:& Fathullah Auzan Setyo Laksono \\
      Nomor Pokok &:& 07211840000053 \\
  
      % Ubah kalimat berikut sesuai dengan semester pengajuan proposal
      Semester &:& Ganjil 2021/2022 \\
  
      % Ubah kalimat-kalimat berikut sesuai dengan nama-nama dosen pembimbing
      Dosen Pembimbing &:& 1. Reza Fuad Rachmadi, S.T., M.T., Ph.D. \\
      & & 2. Dr. Eko Mulyanto Yuniarno, S.T., M.T. \\
  
      % Ubah kalimat berikut sesuai dengan judul tugas akhir
      Judul Tugas Akhir &:& \textbf{Estimasi Umur, Gender dan Etnik Menggunakan} \\
      & & \textbf{Covolutional Neural Network Berbasis Citra Wajah} \\
  
      Uraian Tugas Akhir &:& \\
    \end{tabular}
  \end{adjustwidth}
  
  % Ubah paragraf berikut sesuai dengan uraian dari tugas akhir
  Fitur wajah seperti identifikasi umur, gender dan etnik dapat sangat berguna dalam banyak pengimplementasian ilmu seperti pengamatan visual, diagnosa medis, sistem interaksi komputer manusia, biometric, pengumpulan informasi, penegakan hukum, pemasaran dan banyak lainnya. Dimana sebagian besar data mengenai fitur wajah tersebut masih diambil secara manual melalui survei ataupun pengamatan pada banyak individu. Berdasarkan World Population Clock pada websitenya, di dunia terdapat lebih dari 7 miliar orang yang tersebar di berbagai macam pulau dan benua. Jumlah tersebut masih terus bertambah sampai sekarang. Dimana di setiap benua dan negara tersebut terdapat berbagai karakteristik dan ciri manusia yang berbeda dengan kata lain Etnik yang berbeda-beda. Dengan banyaknya jumlah penduduk dan keberagamannya tersebut, jika data fitur wajah diambil secara manual akan memakan waktu dan tenaga yang banyak. Oleh karena itu perlu dibuat suatu sistem yang dapat mengestimasi umur, gender dan etnik serta mengyimpan penghitungan datanya untuk mempermudah pengumpulan data. Dimana kamera akan menangkap gambar dari seseorang dan dilakukan proses estimasi umur, gender dan etnik yang kemudian datanya disimpan untuk digunakan kedepannya.
  \vspace{1ex}
  
  \begin{flushright}
    % Ubah kalimat berikut sesuai dengan tempat, bulan, dan tahun penulisan
    Surabaya, Desember 2021
  \end{flushright}
  \vspace{1ex}
  
  \begin{center}
  
    \begin{multicols}{2}
  
      Dosen Pembimbing 1
      \vspace{12ex}
  
      % Ubah kalimat-kalimat berikut sesuai dengan nama dan NIP dosen pembimbing pertama
      \underline{[Reza Fuad Rachmadi, S.T., M.T., Ph.D.]} \\
      NIP. 198504032012121000
  
      \columnbreak
  
      Dosen Pembimbing 2
      \vspace{12ex}
  
      % Ubah kalimat-kalimat berikut sesuai dengan nama dan NIP dosen pembimbing kedua
      \underline{[Dr. Eko Mulyanto Yuniarno, S.T., M.T.]} \\
      NIP. 196806011995121000
  
    \end{multicols}
    \vspace{6ex}
  
    Mengetahui, \\
    % Ubah kalimat berikut sesuai dengan jabatan kepala departemen
    Kepala Departemen Teknik Komputer FTEIC - ITS
    \vspace{12ex}
  
    % Ubah kalimat-kalimat berikut sesuai dengan nama dan NIP kepala departemen
    \underline{Dr. Supeno Mardi Susiki Nugroho, S.T., M.T.} \\
    NIP. 197003131995121001
  
  \end{center}