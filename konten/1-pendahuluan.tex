\section{PENDAHULUAN}

\subsection{Latar Belakang}

% Ubah paragraf-paragraf berikut sesuai dengan latar belakang dari tugas akhir
Umur, gender dan etnik merupakan beberapa hal penting dalam wajah yang menentukan bagaimana seorang 
individu berinteraksi sosial. Setiap bahasa di dunia memiliki panggilan kehormatan yang berbeda-beda 
untuk pria dan wanita, perbedaan umur juga dapat menentukan bagaimana seseorang harus bersikap dengan 
orang yang lebih muda ataupun dengan yang lebih tua, sedangkan etnik juga dapat menentukan cara berbahasa 
dan berperilaku pada seseorang. Kebiasaan dan sikap tersebut sebagian besar tergantung pada kemampuan 
seseorang dalam memperkirakan atau mengestimasi individu tersebut melalui penampakan gender, umur dan 
etnik. Dimana identitas, ekspresi, gender, umur dan etnik disebut dengan fitur dalam wajah.
Selain itu, fitur dalam wajah juga sering digunakan di berbagai bidang, seperti di kepolisian untuk 
mencari pelaku tindak kriminal yang mengidentifkasi pelaku dari wajah. diagnosa medis yang menggunakan 
wajah untuk menentukan penanganan yang cocok untuk pasien. Di dunia bisnis fitur wajah juga digunakan 
dalam pembagian target pasar untu lebih meningkatkan proses bisnis.
Hal ini membuat identifikasi umur, gender dan etnik dapat sangat berguna dalam banyak pengimplementasian 
ilmu seperti pengamatan visual, diagnosa medis, sistem interaksi komputer manusia, biometric, pengumpulan 
informasi, penegakan hukum, pemasaran dan banyak lainnya. Dimana sebagian besar data mengenai fitur wajah 
tersebut masih diambil secara manual melalui survei ataupun pengamatan pada banyak individu.
Berdasarkan World Population Clock pada websitenya, di dunia terdapat lebih dari 7 miliar orang yang 
tersebar di berbagai macam pulau dan benua. Jumlah tersebut masih terus bertambah sampai sekarang. 
Dimana di setiap benua dan negara tersebut terdapat berbagai karakteristik dan ciri manusia yang berbeda 
dengan kata lain Etnik yang berbeda-beda. Dengan banyaknya jumlah penduduk dan keberagamannya tersebut, 
jika data fitur wajah diambil secara manual akan memakan waktu dan tenaga yang banyak.


\subsection{Permasalahan}

% Ubah paragraf berikut sesuai dengan permasalahan dari tugas akhir
Pengambilan data terkait fitur wajah terutama umur, gender dan etnis masih dilakukan secara manual yang 
membutuhkan waktu dan tenaga relatif banyak. Oleh karena itu, diperlukan model yang dapat mengestimasi 
umur, gender dan etnik dari individu untuk mempermudah proses pengambilan data.

\subsection{Penelitian Terkait}

% Ubah paragraf berikut sesuai dengan penelitian lain yang terkait dengan tugas akhir
Berberapa penelitian yang telah dilakukan terkait dengan judul Tugas Akhir ini antara lain dilakukan oleh 
A. Garain et al. GRANet A Deep Learning Model for Classification of Age and GenderFrom Facial Images. 
Dimana pada penelitian tersebut mencoba menggunakan beberapa dataset seperti wikipedia age dataset, 
FG-Net, AFAD, AduenceDB dataset dan UTKFace dataset serta menggunakan model yang arsitekturnya seperti 
Residual Attention Network dengan tambahan parameter “Gate” seperti pada Gated Residual Units (GRU’s). 
Yang berhasil melakukan prediksi umur dan gender dengan baik. Namun belum menggunakan pendeteksian etnik. 
Penelitian lainnya dilakukan oleh G. Guo et al. dengan judul Human Age Estimation What is the Influence 
Across Race and Gender yang menggunakan database MORPH-II dengan data gambar wajah sebanyak 55.000. 
Mereka membandingkan hasil estimasi umur antara individu dengan sesama etnik dan dengan yang berbeda 
etnik. Didapatkan tingkat eror yang signifikan pada percobaan estimasi individu  yang  berbeda etnik. 
Kemudian penelitian oleh M. Shin et al. Face Image-Based Age and Gender Estimation with Consideration of 
Ethnic Difference, pada penelitian ini menggunakan CNN dan SVM yang berfungsi memisahkan dua etnik 
menjadi Asia dan Non Asia yang kemudian hasilnya digunakan untuk mendeteksi umur dan gender dari wajah 
yang diberikan. Dihasilkan bahwa pemisahan etnik dapat meningkatkan keakuratan estimasi umur, namun tidak 
berpengaruh pada gender.

\subsection{Gap Penelitian}
Pada penelitian sebelumnya beberapa sudah menggunakan faktor etnis sebagai penentu estimasi umur dan 
gender, namun belum mengestimasikan ras dan juga masih merupakan program yang mendeteksi foto dari input 
yang berasal dari dataset bukan dari kamera secara langsung.

\subsection{Tujuan Penelitian}

% Ubah paragraf berikut sesuai dengan tujuan penelitian dari tugas akhir
Tujuan yang ingin dicapai dari Tugas Akhir ini adalah untuk mengembangkan sebuah model menggunakan 
Convolutional Neural Network yang dapat mengestimasi umur, gender dan etnik yang mempermudah proses 
pengumpulan dan pengambilan data terkait umur, gender dan etnik.