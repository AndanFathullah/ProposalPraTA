\section{TINJAUAN PUSTAKA}

% Ubah konten-konten berikut sesuai dengan isi dari tinjauan pustaka

\subsection{Penyakit Otak}

% Contoh penggunaan referensi dari pustaka
% Newton pernah merumuskan \citep{Newton1687} bahwa \lipsum[8]
% Contoh penggunaan referensi dari persamaan
% Kemudian menjadi persamaan seperti pada persamaan \ref{eq:FirstLaw}.

Penyakit otak merupakan segala bentuk degeneratif, gangguan metabolik, dan infeksi yang dapat mengakibatkan kerusakan pada otak (American Psychological Association Dictionary, 2020).[9] Kerusakan otak dapat mempengaruhi banyak hal seperti ingatan, panca indra, dan bahkan kepribadian seseorang. Kerusakan otak dapat diakibatkan dari berbagai macam faktor seperti penyakit, genetik, atau cedera traumatis. Kerusakan pada otak dibagi menjadi beberapa kategori, yaitu: cedera otak, tumor otak, dan penyakit neurodegenerative (Reed-Guy, L., Han, S., 2018).[10]

Cedera otak sering disebabkan oleh trauma tumpul atau akibat benturan keras oleh benda tumpul di area kepala. Trauma tumpul dapat merusak jaringan otak, neuron, dan saraf. Kerusakan yang disebabkan oleh trauma tumpul dapat mempengaruhi kemampuan otak untuk mengkontrol dan berkomunikasi dengan  bagian tubuh lainnya. Terdapat berbagai jenis cedera otak, seperti: hematomas, blood clots, contusions, cerebral edema, concussions, dan strokes.[10]

Tumor yang terbentuk pada otak terkadang dapat sangat berbahaya, tumor ini disebut dengan primary brain tumors. Sedangkan dalam beberapa kasus terdapat tumor yang muncul pada bagian tubuh lain dan kemudian menyebar ke otak, tumor ini disebut dengan secondary atau metastatic brain tumors. Tumor otak dapat bersifat ganas (kanker) atau jinak (non-kanker). Tumor diklasifikasikan kedalam kelas 1, 2, 3, dan 4, dimana angka yang lebih tinggi menunjukkan tumor yang lebih agresif.[10]

Penyakit neurodegenerative menyebabkan kinerja otak dan saraf semakin memburuk seiring waktu. Penyakit neurodegenerative dapat mengubah kepribadian seseorang atau membuat seseorang kehilangan ingatannya. Penyakit ini juga dapat menghancurkan jaringan dan saraf pada otak. Beberapa penyakit neurodegenerative seperti Alzheimer dapat berkembang seiring dengan bertambahnya usia, sedangkan penyakit neurodegenerative lainnya seperti Tay-Sachs bersifat genetik dan dapat muncul pada usia dini. Penyakit neurodegenerative dapat menyebabkan kerusakaan permanen pada otak dan sampai saat ini belum ditemukan obat yang dapat menyembuhkan penyakit ini.[10]

\subsection{Stroke}
Stroke adalah suatu keadaan dimana aliran darah menuju ke otak terganggu atau terhambat. Apabila darah tidak mengalir ke otak, maka sel-sel otak akan mulai mati. Hal ini dapat menyebabkan kerusakaan permanen seperti cacat atau bahkan dapat menyebabkan kematian. Stroke dibagi menjadi 3 kategori utama, yaitu: Transient Ischemic Attack (TIA), Ischemic Stroke, dan Hemorrhagic Stroke (Nall, R., Han, S., 2018).[11]

Transien Ischemic Attack (TIA) merupakan dapat disebut sebagai ministroke dan merupakan sebuah peringatan bahwa terdapat masalah terhadap aliran darah ke otak. Segala bentuk penyumbatan atau hal yang menghalangi aliran darah ke otak dapat menyebabkan TIA. Penggumpalan darah dan gejala TIA biasanya berlangsung dalam rentang waktu yang singkat.[11]

Ischemic Stroke terjadi ketika terdapat gumpalan darah membuat darah yang membuat darah tidak mengalir menuju otak. Penggumpalan darah sering disebabkan oleh atherosclerosis yang merupakan penumpukan timbunan lemak pada lapisan dalam pembuluh darah. Ischemic Stroke juga dapat terjadi secara embolic, dimana gumpalan darah yang terdapat pada bagian tubuh lain bergerak menuju area otak. Diperkirakan 15\% dari embolic stroke disebabkan oleh kondisi yang disebut atrial fibrillation, dimana jantung berdetak secara tidak teratur. Selain itu juga terdapat thrombotic stroke yang juga merupakan bagian dari ischemic stroke, dimana terdapat penggumpalan darah pada pembuluh darah otak. Tidak seperti TIA, penggumpalan darah yang menyebabkan ischemic stroke tidak akan hilang tanpa pengobatan.[11]

Hemorrhagic Stroke terjadi ketika pembuluh darah di otak robek atau pecah, dan menumpahkan darah ke jaringan sekitarnya. Terdapat tiga jenis utama hemorrhagic stroke, yang pertama adalah aneurysm, yang menyebabkan sebagian pembuluh darah yang melemah menggelembung ke luar dan terkadang pecah. Yang kedua adalah arteriovenosa, yang melibatkan pembuluh darah yang terbentuk secara tidak normal. Dan yang terakhir tekanan darah yang sangat tinggi dapat menyebabkan melemahnya pembuluh darah kecil di otak dan mengakibatkan pendarahan di area otak.[11]

\subsection{Tumor Otak}
Tumor otak adalah pertumbuhan kumpulan sel-sel abnormal pada otak. Tumor otak bisa bersifat ganas (kanker) ataupun jinak (non-kanker). Ketika tumor jinak atau ganas berkembang pada area otak, maka tekanan pada area otak akan semakin meningkat. Hal ini dapat menyebabkan kerusakan pada jaringan otak, atau bahkan dapat menyebabkan kematian (Lights, V., Han, S., 2017).[12]

Tumor pada otak dibedakan kedalam 2 kategori, yaitu kategori primer dan sekunder. Tumor otak primer merupakan pertumbuhan sel-sel abnormal langsung pada area otak, terdapat banyak kasus dimana tumor otak primer termasuk kedalam kategori tumor jinak. Tumor otak sekunder atau metastatic, terjadi ketika sel-sel kanker menyebar ke otak yang berasal dari bagian tubuh lain, seperti paru-paru atau payudara.[12]

Tumor otak primer merupakan sel kanker yang berasal dari otak. Sel kanker ini dapat berkembang melalui sel otak, meninges (membran otak), sel saraf, dan kelenjar. Pada orang dewasa tumor otak yang paling umum ditemukan adalah gliomas dan meningiomas. Gliomas adalah tumor yang berkembang dari sel glial, sedangkan meningiomas adalah tumor yang berkembang dari membran otak.[12]

Tumor otak sekunder merupakan tipe yang paling banyak ditemukan pada kanker otak. Tumor otak sekunder merupakan sel kanker yang berasal dari organ lain dan menuju ke otak, seperti: kanker paru-paru, kanker payudara, kanker ginjal, dan kanker kulit. Tumor otak sekunder selalu berada pada kategori kanker.[12]

\subsection{Alzheimer}
Alzheimer adalah salah satu jenis dari demensia yang mempengaruhi memori, pikiran, dan perilaku. Menurut alzheimer’s association, alzheimer menyumbang 60\%-80\% dari total kasus demensia. Penyakit ini banyak ditemukan pada usia 65 tahun keatas. Sampai saat ini belum ditemukan obat untuk penyakit alzheimer. Alzheimer merupakan penyakit progresif, yang berarti gejala dari penyakit ini akan semakin memburuk seiring dengan berjalannya waktu, dimana penyakit alzheimer dibagi menjadi 7 stage berdasarkan gejala penyakit yang ditemukan (Herndon, J., Legg, T. J., 2019).[13]

\subsection{Magnetic Resonance Imaging (MRI)}
Magnetic Resonance Imaging (MRI) adalah sebuah metode visualisasi medis yang menggunakan medan magnet dan frequency radio untuk dapat menampilkan citra organ ataupun jaringan pada tubuh.[14] MRI yang dilakukan untuk menampilkan citra otak disebut dengan brain MRI atau cranial MRI.

MRI scan berbeda dengan CT scan ataupun X-ray, dimana MRI tidak menggunakan radiasi sama sekali untuk menghasilkan citra. MRI scan menggabungkan hasil scan 2D untuk dapat mendapatkan citra 3D dari struktur organ, sehingga metode ini banyak digunakan untuk mendeteksi kelainan pada struktur kecil otak seperti pituitary gland dan brain stem (Harkin, M., Han, S., 2017).[15] Kondisi abnormal pada otak dapat dideteksi oleh dokter radiologi dengan membandingkan hasil citra MRI dengan citra MRI otak normal.[16]

\subsection{Convolutional Neuranl Network (CNN)}
Pada deep learning, Convolutional Neural Network (CNN) merupakan salah satu kelas pada Artificial Neural Network (ANN), yang banyak digunakan untuk analisa citra.[17] CNN menggunakan operasi matematika, yaitu convolution pada salah satu layernya.[18] CNN merupakan bentuk regularisasi dari multilayer perceptrons (MLP).[19]  MLP merupakan sebuah jaringan yang terhubung penuh, yaitu setiap neuron dalam satu lapisan terhubung dengan semua neuron di lapisan berikutnya.

Arsitektur dari CNN terdiri dari 3 bagian, yaitu: input layer, hidden layers, dan output layer. Pada CNN hidden layers merupakan tempat terjadinya proses konvolusi dengan fungsi aktivasi paling umum adalah ReLu, yang kemudian diikuti pooling layers, dan normalization layers. Output dari convolution layer berupa feature map yang dapat digunakan sebagai dasar pengklasifikasian image.