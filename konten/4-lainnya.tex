\section{HASIL YANG DIHARAPKAN}

\subsection{Hasil yang Diharapkan dari Penelitian}

Penelitian ini diharapkan dapat menghasilkan sebuah sistem yang dapat mengklasifikasikan 4 tipe kondisi otak, yaitu: Normal, Alzheimer, Tumor dan Stroke dengan menggunakan metode 3D CNN yang dapat digunakan untuk membantu dokter radiologi dalam membuat diagnosa.

\subsection{Hasil Pendahuluan}

Sampai saat ini, kami telah \lipsum[16]

\section{RENCANA KERJA}

% Ubah tabel berikut sesuai dengan isi dari rencana kerja
\newcommand{\w}{}
\newcommand{\G}{\cellcolor{gray}}
\begin{table}[h!]
  \begin{tabular}{|p{3.5cm}|c|c|c|c|c|c|c|c|c|c|c|c|c|c|c|c|}

    \hline
    \multirow{2}{*}{Kegiatan} & \multicolumn{16}{|c|}{Minggu} \\
    \cline{2-17} &
    1 & 2 & 3 & 4 & 5 & 6 & 7 & 8 & 9 & 10 & 11 & 12 & 13 & 14 & 15 & 16 \\
    \hline

    % Gunakan \G untuk mengisi sel dan \w untuk mengosongkan sel
    Data Collection &
    \G & \G & \w & \w & \w & \w & \w & \w & \w & \w & \w & \w & \w & \w & \w & \w \\
    \hline

    Data Pre-Processing &
    \w & \G & \G & \G & \w & \w & \w & \w & \w & \w & \w & \w & \w & \w & \w & \w \\
    \hline

    Pembuatan Model &
    \w & \w & \w & \w & \G & \G & \G & \G & \G & \G & \G & \G & \G & \G & \w & \w \\
    \hline

    Training Model &
    \w & \w & \w & \w & \w & \w & \G & \G & \G & \G & \G & \G & \G & \G & \w & \w \\
    \hline
    
    Evaluasi Model &
    \w & \w & \w & \w & \w & \w & \w & \G & \G & \G & \G & \G & \G & \G & \w & \w \\
    \hline
    
    Pembuatan Laporan &
    \w & \w & \w & \w & \w & \w & \w & \w & \w & \w & \w & \w & \w & \w & \G & \G \\
    \hline

  \end{tabular}
\end{table}